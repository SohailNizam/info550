\documentclass[]{article}
\usepackage{lmodern}
\usepackage{amssymb,amsmath}
\usepackage{ifxetex,ifluatex}
\usepackage{fixltx2e} % provides \textsubscript
\ifnum 0\ifxetex 1\fi\ifluatex 1\fi=0 % if pdftex
  \usepackage[T1]{fontenc}
  \usepackage[utf8]{inputenc}
\else % if luatex or xelatex
  \ifxetex
    \usepackage{mathspec}
  \else
    \usepackage{fontspec}
  \fi
  \defaultfontfeatures{Ligatures=TeX,Scale=MatchLowercase}
\fi
% use upquote if available, for straight quotes in verbatim environments
\IfFileExists{upquote.sty}{\usepackage{upquote}}{}
% use microtype if available
\IfFileExists{microtype.sty}{%
\usepackage{microtype}
\UseMicrotypeSet[protrusion]{basicmath} % disable protrusion for tt fonts
}{}
\usepackage[margin=1in]{geometry}
\usepackage{hyperref}
\hypersetup{unicode=true,
            pdftitle={Homework 2},
            pdfborder={0 0 0},
            breaklinks=true}
\urlstyle{same}  % don't use monospace font for urls
\usepackage{graphicx,grffile}
\makeatletter
\def\maxwidth{\ifdim\Gin@nat@width>\linewidth\linewidth\else\Gin@nat@width\fi}
\def\maxheight{\ifdim\Gin@nat@height>\textheight\textheight\else\Gin@nat@height\fi}
\makeatother
% Scale images if necessary, so that they will not overflow the page
% margins by default, and it is still possible to overwrite the defaults
% using explicit options in \includegraphics[width, height, ...]{}
\setkeys{Gin}{width=\maxwidth,height=\maxheight,keepaspectratio}
\IfFileExists{parskip.sty}{%
\usepackage{parskip}
}{% else
\setlength{\parindent}{0pt}
\setlength{\parskip}{6pt plus 2pt minus 1pt}
}
\setlength{\emergencystretch}{3em}  % prevent overfull lines
\providecommand{\tightlist}{%
  \setlength{\itemsep}{0pt}\setlength{\parskip}{0pt}}
\setcounter{secnumdepth}{0}
% Redefines (sub)paragraphs to behave more like sections
\ifx\paragraph\undefined\else
\let\oldparagraph\paragraph
\renewcommand{\paragraph}[1]{\oldparagraph{#1}\mbox{}}
\fi
\ifx\subparagraph\undefined\else
\let\oldsubparagraph\subparagraph
\renewcommand{\subparagraph}[1]{\oldsubparagraph{#1}\mbox{}}
\fi
\usepackage{hyperref}
\hypersetup{colorlinks=true,urlcolor=blue}

\title{Homework 2}
\author{}
\date{\vspace{-2.5em}}

\begin{document}
\maketitle

Nothing required to be turned in for question 1. For question 2, submit
your R code as plain text. For question 3, submit your bash code for
breakout exercises 1, 2, and 3 (if asynchronous) or submit your grade
for each student's participation in your breakout group \textbf{for
class on 2020/09/03} (if synchronous) in the following format:

\begin{itemize}
\tightlist
\item
  Susy Student = 1
\item
  Cam Classmate = 1
\item
  Danny Scientist = 0
\end{itemize}

\begin{enumerate}
\def\labelenumi{\arabic{enumi}.}
\tightlist
\item
  Install the following software:
\end{enumerate}

\begin{itemize}
\tightlist
\item
  if on a Mac, install
  \href{http://railsapps.github.io/xcode-command-line-tools.html}{Xcode
  command line tools}

  \begin{itemize}
  \tightlist
  \item
    confirm proper installation by checking \texttt{which\ make}
  \end{itemize}
\item
  \href{https://cran.r-project.org/web/packages/rmarkdown/index.html}{\texttt{rmarkdown}}
  R package
\item
  \href{https://pandoc.org/installing.html}{\texttt{pandoc}}, using
  homebrew, and ensure that the directory containing \texttt{pandoc} is
  in your \texttt{PATH}.

  \begin{itemize}
  \tightlist
  \item
    confirm by checking \texttt{which\ pandoc}
  \end{itemize}
\end{itemize}

\begin{enumerate}
\def\labelenumi{\arabic{enumi}.}
\setcounter{enumi}{1}
\tightlist
\item
  Write an R script that begins to analyze the data for your project.
  For now the script could do very simple things. The only requirements
  for this assignment are that you read in data and \texttt{print} out
  something about the data. For example, the script could
\end{enumerate}

\begin{itemize}
\tightlist
\item
  read in your data, and \texttt{print} the number of observations and
  the number of missing values for each variable
\item
  read in your data, perform some data cleaning, save a cleaned data
  set, \texttt{print} a message indicating where the file was saved.
\end{itemize}

The point of this problem is not \emph{what} the R script does, just
that it \emph{does something} and prints something. However, it would
likely be beneficial to start writing a script that does
\emph{something} that will eventually be used by your report.

Once your R script is written:

\begin{itemize}
\tightlist
\item
  add the appropriate shebang to use the \texttt{Rscript} executable to
  run your script from the command line;
\item
  make your script executable using \texttt{chmod};
\item
  execute your script from the command line and confirm it executes as
  expected;
\item
  execute your script from the command line redirecting output of the
  script to a file called \texttt{script\_out.Rout}.
\end{itemize}

\begin{enumerate}
\def\labelenumi{\arabic{enumi}.}
\setcounter{enumi}{2}
\tightlist
\item
  If you attended class synchronously, give participation grades to your
  classmates (0 = not present/did not say anything, 1 = present). If you
  attended class asynchronously, include your code for each of the
  breakout exercises.
\end{enumerate}

\end{document}
